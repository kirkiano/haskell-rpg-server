
Communication is serialized in JSON\@. The objects defined
in \secref{play-formal} obviously involve redundancy of types. For
example, to have defined \youarevalset\ as
$\{ \youaretg \} \times \nameset $, eliding the technically
unnecessary constant \charnametg, would have produced serializations
that are leaner. But we have opted against such premature optimization
of payload size, at least for now, because stronger typing helps
prevent errors during {\em de}serialization. For example,

For example, if a character wishes to say ``Good morning'' to other
characters in the same place, it constructs an instance of
the \typ{Say} type, which is a subtype of the \typ{Request} type that
holds the string ``Good morning,'' and sends it to the server.  The
corresponding client thread pairs the request with this client's
unique ID, and pushes the pair onto the game loop's in-queue.  The
game loop eventually pops the request off and, seeing that it is of
type \typ{Say}, it~queries the database for the set of characters
within ``earshot'' of the requesting client (\ie, those occupying the
same place).  It~then constructs an instance of the
\typ{Said} type, which is a subtype of the \typ{Event} type, and wraps it up
as instance of type \typ{Message}.
To forward this message to the other characters, it uses their sending
functions, which it looks up in its internal dictionary.

