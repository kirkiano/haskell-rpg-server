
Currently the disk-resident database is Postgres, and this is not likely to
change, as the relational model suits the game naturally.
An earlier version of the engine interposed
Redis as a cache, but apparently Postgres does its own caching so well that on
average Redis cannot accelerate queries by more than a factor of two.\footnote{
See \url{https://thebuild.com/blog/2015/10/30/dont-assume-postgresql-is-slow/}
and \url{http://code.jjb.cc/benchmarking-postgres-vs-redis}.
But see
also the conflicting report mentioned below.} Roughly the same is presumably
true of the other well known on-disk databases, \eg, MySQL\@.
\S\ref{subsec:future:impl} includes a discussion of ways to make
the engine faster.

Currently the game loop waits for a database update to complete just as it
waits for a query to return data.
There is therefore no point in pooling multiple connections, as the game loop
would use only one of them at a time.
Another recommendation against pooling
is given below.
