A {\em place} is the game's smallest geographical unit; it is similar
to a chat room. The set of all possible places is denoted \placeset{},
and for each $\sigma \in \Sigma$ we define the set
\[ \placeset{\sigma} \subset \placeset{} \]
of places valid (\ie, present) in~$\sigma$.
The following families of
state functions define the properties of a place.

%% The subset of places valid (\ie, present) in a state is given
%% by
%% \[ \placessym:~ \Sigma \rightarrow \powset{\placeset} \]
A place {\em may} be a part of an address:
\[ \placeaddresssym{\sigma}: \placeset{\sigma} \surj \ornothing{\addressset{\sigma}} \]
Surjectivity here reflects the rule that each address
must have at least one place assigned to it.
The name of a place is given by\footnote{The actual implementation
does not require uniqueness of place names, but of the combination of
a place's name and its address, if any.}
\[ \placenamesym{\sigma}:~ \placeset{\sigma} \inj \placenameset\]
and its (not necessarily unique) description is given by
\[ \placedescsym{\sigma}:~\placeset{\sigma} \rightarrow \placedescset \]
%% Both must be defined for all valid places:
%% \begin{align*}
%% \forall \sigma, pid
%% ~.~~ & \placename{\sigma}{pid} = \nothing & ~~~~\text{iff}~~~~ &
%% pid \not\in \placeids{\sigma} \\
%% \forall \sigma, pid
%% ~.~~ & \placedesc{\sigma}{pid} = \nothing & ~~~~\text{iff}~~~~ &
%% pid \not\in \placeids{\sigma}
%% \end{align*}
