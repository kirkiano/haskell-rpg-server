The requests that a player can make of the game are as follows:
\begin{itemize}
\item \whoamireqset. Send this request, which is just the constant
  \whoamitg, to learn about your own character:
  \[ \whoamireqset \equiv \{ \whoamitg \} \]
The expected response is a value from \set{YouAre}
(\secref{play-formal:values}).
\item \whereamireqset. Send this request, which is just the constant
  \whereamitg, to learn about your current location:
  \[ \whereamireqset \equiv \{ \whereamitg \} \]
The expected response is a value from \set{Place}
(\secref{play-formal:values}).
\item \whatisherereqset. Send this request, which is just the constant
  \whatisheretg, to receive a list of things present in your current
  location:
  \[ \whatisherereqset \equiv \{ \whatisheretg \} \]
The expected response is a value from \set{PlaceContents}
(\secref{play-formal:values}).
\item \quitreqset. Send this request, which is just the constant
  \quittg, to tell the game that you are quitting:
  \[ \quitreqset \equiv \{ \quittg \} \]
\end{itemize}
The set of all possible requests is the union of the above sets:
\begin{equation}
  \reqset \equiv \whoamireqset ~\disjunion~ \whereamireqset
  ~\disjunion~ \quitreqset
\end{equation}
