We settle a few points of notation before proceeding. The power set of
a set $S$ is denoted \powset{S}.  The domain and codomain of a
function $f: A \rightarrow B$ are respectively denoted
\begin{align*}
  \domain{f} & = A\\
  \codomain{f} & = B
\end{align*}
An injective function $f$ is denoted $f: A \inj B$,
a surjective one $f: A \surj B$.

We freely use currying, uncurrying, and partial application. For
example, if $f: A \times B \rightarrow C$, then we sometimes use the
same letter $f$ to denote the curried function $f: A \rightarrow
B \rightarrow C$, and vice versa.

The null value is denoted by \nothingsym, and for any set $S$ we define
$\ornothing{S} \equiv S \cup \{\nothingsym\}$. The notation $f: D
\rightarrow \ornothing{R}$ means that the function $f$ is partial on
$D$. A function $f: A \inj \ornothing{B}$ is to be understood as
injective only on $f^{-1}(B)$ (unless $ \nothingsym \in B$). Similarly,
$g: A \surj \ornothing{B}$ is to be understood as surjective only on
$g^{-1}(B)$ (unless $\nothingsym \in B$).  Unless noted otherwise, all
functions are strict in \nothingsym.

The {\em singleton operator} wraps an element of a set:
\[ \singletonopsym{A}: \,A \inj \powset{A}, ~~~~~~~~~~
\singletonop{A}{x} \equiv \{ x \} \]
And it is convenient to define the {\em membership relation}
restricted to a set $A$:
\[ \memopsym{A} = \{ (x, S) ~|~ \{ x \} \subseteq S ~~\text{and}~~ \{
x \} \subseteq A \} \]
