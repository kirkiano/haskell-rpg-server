
Communication is serialized in JSON\@. The objects defined
in \secref{play} obviously involve redundancy of types. For
example, to have defined \youarevalset\ as
$\{ \youaretg \} \times \nameset $, eliding the technically
unnecessary constant \charnametg, would have produced serializations
that are leaner. But we have opted against such premature optimization
of payload size, at least for now, because stronger typing helps
prevent errors during {\em de}serialization.

%% For example, if a character wishes to say ``Good morning'' to other
%% characters in the same place, it constructs an element of
%% \sayreqset (a subtype of \reqset) that
%% holds the string ``Good morning,'' and sends it to the server.  The
%% corresponding client thread pairs the request with this client's
%% unique ID, and pushes the pair onto the game loop's in-queue.  The
%% game loop eventually pops the request off and, seeing that it is of
%% type \sayreqset, it~queries the database for the set of characters
%% within ``earshot'' of the requesting client (\ie, those occupying the
%% same place).  It~then constructs an instance of the
%% \saidevtset\ type (a subtype of \evtset).
%% To forward this message to the other characters, it uses their sending
%% functions, which it looks up in its internal dictionary.

If JSON should ever become a bottleneck, the option to communicate in
raw binary can be added to the game engine.\footnote{But if
communication is dominated by long strings of text then this will not
bring a big improvement, because in strings there is already a roughly
one-to-one correspondence between character and byte.}
