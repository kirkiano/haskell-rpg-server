The model
of \S\S~\ref{sec:play:states:place}~and~\ref{sec:play:states:address}
forms places into a hierarchical tree. The root is implicitly the
Earth, places with no address have depth 1, and those with addresses
have depth 5 (countries have depth 1, cities have depth 2, streets
have depth 3, and addresses have depth 4). That this model could be
replaced by a more flexible one can be seen in its inability to
represent overlap. For example, if two countries are separated by a
lake, then the current model gives no way of partitioning the lake
into (at least) three parts, one that falls within one of the
country's borders, one that falls within the other's, and a third that
is ``international waters.'' A better model would recognize that the
three parts of the lake all belong to the same body of water, but that
two of them also form parts of countries. Such a model would be more
true to life and would improve the gaming experience. Similarly, if
buildings should be introduced to the game, then under the current
model they would have to be made children of addresses, and there
would be no way to represent, say, a building that sits in some
country but has no formal address, such as a Mayan temple.

These limitations can be removed if the tree model were replaced by a
dag (directed acyclic graph). As with the tree, the places that a
player can actually inhabit remain the leaves (where a leaf in the dag
context is understood to be a node with no out-edges), but they, along
with their ancestors, can have more than one parent. This solves the
problem of the lake that straddles national boundaries. As for the
temple, not all buildings in the dag model need have addresses as
parents.

The question then arises as to the order in which a place's ancestors
should be listed. In a tree, the ancestors of a node form a single
path to the root, which induces them into a natural linearly ordering
(\eg,~a~room in a building on an addressed property within a city of a
country). In a dag, the ancestors form a sub-dag which would have to
be topologically sorted. Since some of the ancestors will be
incomparable under the induced ordering (\eg,~should the lake be
listed before the country, or vice versa?), additional rankings
will be needed to arrange forcanonical order.

Finally, implementation notes are in \secref{future:impl:place_dags}.
