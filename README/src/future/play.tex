\subsubsection{Police, journalists, corruption, and surveillance}

A driver can supply the game with NPCs representing
the police, or various authorities.
Such authorities can become corrupt, making the game quite interesting.
They can, for example, victimize a character by planting evidence on it,
perhaps to punish it for having blown the whistle on antecedent corruption,
or for having practiced (real) journalism.

Similarly, another driver can simulate researchers who publish journal papers
which are nonsense but nevertheless peer-reviewed, and which the court system
(simulated by yet another driver) can use, perhaps secretly, to justify
nefarious police practices.
For example they can target
an innocent character for covert defamation or harassment, wait for
the target's life to begin falling apart (see the section on Economics below
and its discussion of earning a living), wait a little more for the target to
react with the predictable rage and despair, use that reaction to label the
target as "dangerous," and then use that "finding" to justify the targeting
in hindsight, or even to justify an escalation of the torment.
Such self-fulfilling dynamics could
rely on the nonverbal mechanisms mentioned above, and could also use
surveillance to violate the character's privacy.

If a character speaks out publicly about such targeting without being able to
prove it, then the state should appoint a mental health professional to falsely
certify the victim as a lunatic, arrest it, and confine it forcibly to a prison
or a psychiatric asylum.

\subsubsection{Virtuous infiltrations}

A player (live or NPC) might try to counteract such corruptions by, say,
signing up to become a police officer and trying anonymously inform the media
of such corruption.
Officers of a different kind might try to fight fire with fire,
by planting incriminating evidence on corrupt officers, giving them a taste
of their own medicine.
Whatever mechanisms are introduced into the game,
it will be interesting to see if corruption can be cured.

\subsubsection{Aggregate measures}

Aggregations of individual health levels can be used to assess the well-being
of whole populations.
If a character is not isolated by ostracism, it should work to keep both the
individual and the community in good health during wars, physical or
psychological.

\subsubsection{Emergence and phase transitions}

Political health can be measured by other aggregates, such as the fraction of
the population engaged in factionalism, or in vigilantism on behalf of police,
or in vigilantism {\em against} the police, or the fraction that believe the
media gullibly, \etc\@
Out of a given set of parameters (rate of corruption, rate of gun ownership,
rate at which gossip is transmitted, sensitivity of characters to
psychological effects) will emerge a certain dynamics, which
in time will presumably equilibrate to a certain {\em phase}, to borrow the
language of statistical mechanics.
It will be interesting to measure the corresponding relaxation
times, and to attempt phase transitions by tweaking the "macroscopic"
parameters just mentioned.

\subsubsection{Machine learning}

A driver can use machine learning
to guide the behaviors of its NPCs. For example such algorithms
might be used to improve the aim of an NPC as it practices at the firing range.

Another example: the reference above to journalism suggests a driver
that supplies a media apparatus.
That is, NPCs can be made to consume news
reports, be they true or propagandistic. (It is not yet clear whether
such a requirement would have to be enforced by the game engine.) Perhaps
machine learning, or something more elementary, can be used to shape their
reactions to such reports.
They may, for example, make them more gullible,
ie, more trusting of authorities.
That in turn may make them more inclined
to act as vigilantes on behalf of authorities, and to harass one of the
innocent victims mentioned above.

\subsubsection{Self-fulfilling loopholes}

The ``legislature'' of such a society could pass ``laws'' (computable rules)
permitting a certain class of outrageous acts, such as, for example, putting
a child or an elderly person to death if he or she becomes burdensome in
some way.
This of course would make some individuals angry.
Therefore, the legislature could concomitantly pass a law criminalizing anger,
or at least anger about such an issue, so that any person who evinces the
anger in any way is to be formally classified as dangerous.
Having inflicted that classification, the police can then go about
harassing and molesting such individuals under the guise of
surveillance.\todo{Cite the Soho Forum debate of August or September 2019,
in which the philosophy professor says that any person deemed dangerous should
be subjected to ``surveillance.''}
This of course sets up a feedback loop that will eventually provoke a certain
fraction of the targets to violence, and the outbursts are to be taken as
post-hoc proof that the violent individuals really were violent to begin with.
Such macabre self-fulfilling prophecies should make the game an interesting
challenge, as long as parameters are turned so as not to make the outburst
rate too high.