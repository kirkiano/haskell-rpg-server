Each character should have a set of {\em feelings}: fear, happiness,
anger, pride, humiliation, \etc, each with its own scale.  As in real
life, emotions and biological reactions should happen {\em to} a
character, whether he or she wills it or not, so various looks and
gestures (\S\ref{sec:future:rules:nonverbal}) should trigger the
consequent psychological effects, positive and negative.  That is,
they should influence the receiver's feelings.  For example, friendly
looks and gestures (if they are sincere) should increase a character's
psychic well-being.  Similarly, the wrong kind of look can have bad
effects: they can provoke hysterical fear in some primates, and combat
in geese.  These dynamics should be part of the game's mechanics
regardless of characters' higher-level gameplay(s).

Conversely, if certain emotions become too intense, they should make a
character react in certain ways.  If enough physical or psychological
torment is inflicted on it, then the humiliation should eventually
push its level of rage beyond the limit of self-control, and it should
act out in some way, perhaps with physical or emotional violence, eg,
a slap or a scream.  It will be interesting to see the effect of such
surprises on live players.  On NPCs they should provoke further
reactions, positive or negative depending on the particular gameplay.
In this way the mechanics of the game should give rise to feedback
loops.

These two classes of measure---the physical and the
psychological---should affect one another.
For example one has heard of people, and animals, dying of grief, and
even of joy.
And stress is generally thought to raise the likelihood of heart
attack. Reactions should be tuned so that they do not lead to chains
that ``explode'' too quickly.

Notes for the implementation of these features are given in
\S\ref{sec:future:impl:psycho}.
