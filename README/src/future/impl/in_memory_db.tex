As a matter of interest we can ask whether persistence can be made even faster.
Apparently, in addition to making transactions fast, Redis also makes them
durable, by keeping a transaction log on disk, writing to it on every update of
in-memory data, and making the write operation an append.
Since the I/O entailed in appends is sequential, there is no seek delay,
making the updates fast.
Details can be found
\footlink{here}{https://medium.com/@denisanikin/what-an-in-memory-database-is-and-how-it-persists-data-efficiently-f43868cff4c1}.
(A \footlink{companion article}{https://medium.com/@denisanikin/asynchronous-processing-with-in-memory-databases-or-how-to-handle-one-million-transactions-per-36a4c01fc4e4}
raves about the speeds attainable with Redis.
But another online post, which I now cannot find, measured the just-described
``embedded'' approach as being orders of magnitude faster than Redis.)

The game engine could do the same.
That is, after successfully servicing a request,
the game loop could immediately write it to the end of a transaction
log file on disk.
If and when the size of this log file becomes unwieldy,
it can be wiped clean, and the in-memory structure representing the current
state of the game can be saved to disk instead.
From this fresh checkpoint the transaction log, having just been truncated,
could then build up again.
And the cycle would continue.

But this trick would seem not to work as well if the hard drive were being
shared with other processes performing I/O. Moreover it would mean the
abandoning of the relational model, at least near the game engine, and hence
the loss of a clean data model which could also be used by ancillary apps.
So it is hoped that the needs of speed will not require this drastic change.
According to
\footlink{Postgres' documentation}{https://www.postgresql.org/docs/9.1/explicit-locking.html},
``An exclusive row-level lock on a specific row is automatically acquired when
the row is updated or deleted.'' This of course increases the throughput of
transactions, and we hope it will make them plenty fast.
