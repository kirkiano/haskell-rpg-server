The function of the game engine is to multiplex incoming game requests,
send them to the game loop, and let it process them and return to the
respective requesters (clients) messages representing either requested data
or notifications that some game event has occurred.
The server assigns each client socket its own thread and gives it access to the
game loop's in-queue, onto which the client pushes its requests.
The game loop, running in its own thread, pops the requests
off this master queue and services them in order, sending a query or an update
to the database if necessary.

For example, if a character wishes to say ``Good morning'' to other characters
in the same place, it constructs an instance of the \typ{Say} type, which is a
subtype of the \typ{Request} type that holds the string ``Good morning,'' and
sends it to the server.
The corresponding client thread pairs the request with this client's unique
ID, and pushes the pair onto the game loop's in-queue.
The game loop eventually pops the request off and, seeing that it is
of type \typ{Say}, it~queries the database
for the set of characters within ``earshot'' of the requesting client (\ie,
those occupying the same place).
It~then constructs an instance of the
\typ{Said} type, which is a subtype of the \typ{Event} type, and wraps it up
as instance of type \typ{Message}.
To forward this message to the other characters, it looks up
their client sockets in a master dictionary, and sends it to each.

Communication is serialized in JSON\@.

The engine is implemented in Haskell and imports helper libraries from other
repos.
The advantages of Haskell include
garbage collection, green threads, strict immutability, algebraic data types,
type inference, a parametric type system that includes ad hoc polymorphism plus
constraints, lazy evaluation and equational reasoning, a surpassingly terse
and elegant syntax, libraries rich in convenient abstractions, and, of course,
portability.
The disadvantages of Haskell include the lags of
garbage collection, the profligate use of memory by a lazy
language, and an inability to match the speeds of a low-level language such as
C, at least for most kinds of computation (though Haskell does do
\footlink{better than
the JVM}{https://benchmarksgame-team.pages.debian.net/benchmarksgame/fastest/haskell.html} on several measures).
The possibility of using other languages and runtimes is discussed below.

\todo{Say something about Kafka.}

\subsubsection{Things, Characters, and Places}

Things are instances of type \typ{Thing}, which include a name,
a unique ID,\footnote{Unique only among things.}
and
possibly a description.
The subtype \typ{Character} represents characters, though at this time there
is almost no difference between instances of the two.
Places are instances of type \typ{Place}, which bears a name,
a unique ID,\footnote{Unique only among places.}
possibly a description, and possibly also an address.
As in real life, an address serves as a tag for a set of places that together
form a larger location.
For example each of the abovementioned scraping bots is currently confined to
a specific address, so that it can move only between the places within that
address.

Currently the game engine does not distinguish between a live player and a
non-player character (NPC).
The words ``player'' and
``character'' will typically be used to denote live players and
NPCs indifferently, but sometimes ``player'' will mean a live player\,;
context should make the meaning clear.
Similarly, ``thing'' will be used to
mean either an inanimate object, an NPC, or a live player.

\subsubsection{Persistence}

Currently the disk-resident database is Postgres, and this is not likely to
change, as the relational model suits the game naturally.
An earlier version of the engine interposed
Redis as a cache, but apparently Postgres does its own caching so well that on
average Redis cannot accelerate queries by more than a factor of two.\footnote{
See \url{https://thebuild.com/blog/2015/10/30/dont-assume-postgresql-is-slow/}
and \url{http://code.jjb.cc/benchmarking-postgres-vs-redis}.
But see
also the conflicting report mentioned below.} Roughly the same is presumably
true of the other well known on-disk databases, \eg, MySQL\@.
\S\ref{subsec:future:impl} includes a discussion of ways to make
the engine faster.

Currently the game loop waits for a database update to complete just as it
waits for a query to return data.
There is therefore no point in pooling multiple connections, as the game loop
would use only one of them at a time.
Another recommendation against pooling
is given below.

\subsubsection{Bots}\label{sec:bots-current}

The bots are implemented in Python (a separate repo) and connect to the
engine over TCP\@.
But this way of providing NPCs will not scale up well to a large number
of them, because it would make the server lose too much time context-swtiching
between their connections.
(And even if the server were implemented in a language that favored running
tasks asynchronously in an event loop, as is envisioned in
\S\ref{subsec:future:impl}, this approach still consumes
too many kernel resources.)
A better design is therefore planned in \S\ref{sec:future:impl:drivers}.