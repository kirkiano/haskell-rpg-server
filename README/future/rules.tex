\subsubsection{Nonverbal gestures}
\label{sec:future:rules:nonverbal}

The looks that a character can currently give another are neutral.
Game mechanics should include a host of others that are pregnant with meaning:
smiles, frowns, smirks, glowers, looks of shock, pity, joy,
curiosity, confusion, upset,
\etc\@ As in real life, characters should give each other neutral looks
randomly and at some low frequency.

Similarly there should be gestures of the body, and the game should report
(as events) whether a character's body language is not neutral or natural, \eg,
whether it is stiff with pretense, or slumped with fear, or erect with contempt.

\subsubsection{Health, physical and psychological}

As in many other games, each character should have its own measure of health.
But this should be broken down by type, giving a measure of heart rate,
blood pressure, rate of breath, \etc

More interestingly, each character should have a set of {\em feelings}:
fear, happiness, anger, pride, humiliation, \etc, each with its own scale.
As in real life, emotions and biological reactions should happen {\em to} a
character, whether he or she wills it or not, so various looks and
gestures (\S\ref{sec:future:rules:nonverbal}) should trigger the consequent
psychological effects, positive and negative.
That is, they should influence the receiver's feelings.
For example, friendly looks and gestures (if they are sincere) should increase
a character's psychic well-being.
Similarly, the wrong kind of look can have bad effects: they can provoke
hysterical fear in some primates, and combat in geese.
These dynamics should be part of the game's mechanics regardless
of characters' higher-level gameplay(s).

Conversely, if certain emotions become too intense, they should make a
character react in certain ways.
If enough physical or psychological torment is inflicted on it, then the
humiliation should eventually push its level of rage beyond the limit of
self-control, and it should act out in some way, perhaps with physical or
emotional violence, eg, a slap or a scream.
It will be interesting to see the effect of such surprises on live players.
On NPCs they should provoke further reactions,
positive or negative depending on the particular gameplay.
In this way the mechanics of the game should give rise to feedback loops.

These two classes of measure---the physical and the psychological---should
affect one another.
For example one has heard of people, and animals, dying of grief, and even
of joy.
And stress is generally thought to raise the likelihood of heart attack.
Reactions should be tuned so that they do not lead to chains that ``explode''
too quickly.

Comments on the implementation of these features are given in
\S\ref{sec:future:impl:psycho}.

\subsubsection{Gossip}

To mimic the less savory aspects of real life, characters should be able to
whisper defamatory gossip about one another.
And as in real life, unfortunately, once character A hears gossip about
character B, A should not try to verify the claims, but rather should change
its nonverbal behavior toward B\@, giving it looks and gestures that
reflect the bad-mouthing (\S\ref{sec:future:rules:nonverbal}).
This change in attitude should be permanent, and once a character hears gossip
about another, it should spread it to others, depending on its severity.

For implementation details see \S\ref{sec:future:impl:gossip}.

\subsubsection{Property and privacy}

When a character disconnects from the server, the engine should automatically
put it back in its home place while it remains offline.

A permission system should be established, allowing characters to own
property and to forbid others from entering.
In this way a character's home can be made a refuge from whatever wars,
physical or psychological, or going on outside.

\subsubsection{Messaging}

The game should have a mechanism that allows a character to send a message,
such as a letter, to another one.
When character A receives a message in the presence of character B
(\ie, if they are both in the same place at that time), the game
should inform B of~it, even if the message does not regard~B\@.

\subsubsection{Weapons}

Elements of more traditional warfare can be introduced in the form of guns,
\etc\@
A character would be able to acquire a gun and shoot at a target, including
an enemy character.
But whether the shot hits is determined by the character's skill.
The game could require characters to practice at firing
ranges, as a condition for improving the accuracy of their aim.
To mimic real life, when it comes time to fire at an actual enemy target later
on, a probability, computed as a function of the extent of past practice, would
determine whether the shot is successful.

\subsubsection{View port}

The simulation of firing ranges raises the question of whether the central
component of the game app should be a graphical view port, rather than a
chat stream, as is currently the case in the Angular frontend.
The game is committed to written language as the medium of communication,
especially as high-resolution animated graphics, even if quickly
rendered, can sometimes amount to bells and whistles that distract from the
essence of the game.
But here and there it might be nice to have an image to look at, if only one
that depicts the current location.

\subsubsection{Economics}

The game engine can incorporate a monetary system and an economy in which
characters trade various goods and services as they try to eke out
an existence.
Driving modules can be added representing various professions;
journalism and policing have already been mentioned.
Agriculture is another possibility: fields for planting can be added,
and a driving module can fill them with farmers.
The game engine may require characters to eat some kind of food if they are
to maintain their health.

And a character can be required to earn a living, using the wages to
purchase various necessities, including food and weapons.
The ability to do this comfortably can be impaired as part of an enemy's attack
on the character, for example by harming the character's reputation.
