To implement the involuntary reactions mentioned above, the game engine will
have to react to its own `Event`s.
One way to arrange for this is to define a
function \co{react :: [Event] -> m [Request]}, which computes `Request`s for
reactions, based on, say, statistics about characters' histories, and which
the game loop could invoke on each `Message` it has just generated, and whose
resulting `Request` (if any) it would push onto its own in-queue, to be
processed shortly thereafter.

For example, say character A has just punched character B\@.
The game loop receives A's `Request` for the punch, `respond` produces
the `Event` indicating that the punch has happened, and the
game loop sends this `Event` (via the forwarder) to any other players
in the vicinity.
But before the game loop turns to the next incoming `Request`
on its queue, it calls `react` on this "has-punched" `Event`.
The `react`
function examines the details of the character's current biological and
psychological state, and decides, deterministically or otherwise, that
character B is to suffer an increase in heart rate and should fly into a
screaming rage.
Accordingly it returns a list of two `Request`s corresponding to these
reactions.
The game loop then pushes them onto its own in-queue.
A few iterations later, these `Request`s are up for processing, and from them
the `respond` function of course generates two `Event`s indicating that the
character's heart rate has increased and that an outburst is happening (the
former might be sent only to the character in question, whereas the latter,
being a visible outburst, would need to be sent to nearby characters as well).
But again, before moving on, the game loop should invoke `react` on these two
`Event`s as well.
Supposing that `react` finds that B's rage has
crossed a certain threshold, it might produce a `Request` to raise B's blood
pressure.
Again the game loop enqueues that `Request` on its
own in-queue.
And so the game loop becomes a feedback loop, and `react` decides
whether the feedback is positive or negative, depending on the circumstances.

This is one way of implementing such involuntary reactions.
Note that it risks generating reaction `Event`s at too high a rate, perhaps
even an exponential one.
One way to limit such growth is to ensure that the game loop calls `react`
no more than once per iteration, and to limit `react` to returning no more than
one `Request`, making its signature `[Event] -> Maybe Request`.

Another way of implementing the feature is to outsource the `react` function
to its own driving module.
The game loop would then send input `Event`s to
`react` to this module via asychronous channel, and the module would return any
resulting `Request` to the game engine, which would push it onto the master
in-queue as it does for any other client.
The advantage of this design is that
the driving module could be written in the kind of higher-level language that
is more typically used for such artifical intelligence: eg Lisp, Scheme,
Prolog.
If it is to react to things said by one character to another, then it
might also use one of the Python libraries for natural language processing.

But the approach also has disadvantages.
For one thing, the languages just mentioned have implementations that are
slower than GHC, and much slower than Rust.
So the module's
pace may not be able to match that of the game engine, and involuntary
reactions may become so delayed as to lose correlation with their causes.
That is, they would no longer make psychological sense.
Delays may also arise because the `react` function, whatever language it is
implemented in,
may not be able to make certain decisions without first getting data that
can only be gotten from the database, in which case it would either have to
query the DB directly or\ldots\ send an intermediate `Request` to the game loop.

The only way to match the paces is to make the game loop wait for `react`,
ie, to make the call block.
This of course would slow down the whole game,
especially if the driving module used sluggish libraries from the languages
just mentioned.
The call can be sped up by not using a separate driving module
and implementing `react` in the game engine itself, bringing us back to the
beginning of this discussion.
Since this would mean writing `react` in
the same language as the rest of the game engine (or in a faster one via FFI),
this point effectively becomes a vote against writing the game engine
in a fast language such as Rust.
As impressive as Rust is, it does not seem suited to the higher-level logics of
artificial intelligence.
(\footlink{This Quora post}{https://www.quora.com/Is-Rust-a-good-language-for-artificial-intelligence}
makes the point with more nuance.) But it can still be used to model the more
biological reactions of `react`, ie, those that entail just a bit of arithmetic,
and it can still be called by Haskell via FFI\@.