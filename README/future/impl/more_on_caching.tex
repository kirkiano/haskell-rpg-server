Because most messages to clients are limited to carrying names and IDs
(which are short) and not descriptions (which can be long), it may be tempting
keep the descriptions out of the cache (thus sparing memory) and/or to spare the
game loop the work of sending them, by providing them instead via side-channel
REST API (exposed by Django, say). Unfortunately this won't work, because
gameplay requires that whenever a character looks at another Thing, the other
characters in the same place must be immediately notified of the look, which
of course is most easily done by the game loop.
(Future gameplay may require similar notifications when
it is the current Place that is being looked at.) Still, the size of the cache
can be managed by purging it of data that is no longer used. For example, if
a player is the last to leave a Place, then the cache can be purged of the
place's details (name, ID, description, and exits) and of any inanimate Things
in it.

The question then is how to maintain consistency with the on-disk DB. One
solution is to take the DB connection out of the game loop and to entrust
it instead to a separate thread. The game loop would then effect updates
to the DB only by sending requests via this cooperating thread. (If possible,
make these requests consist simply of the Haskell IO actions representing DB
updates.)

For consistency the cooperating thread should access the DB only
by a single connection, and not a pool. Suppose for example that two
characters, A and B, are in place 1, and a third, C, is in the adjoining
place 2. Character A sends the engine a request to move to place 2, and then
it sends another to move back to place 1. The TCP connection from the client
will ensure that these requests arrive at the game engine in order, and they
will therefore remain in order when the thread corresponding to player A pushes
them onto the game queue. The game loop in turn will process them in order.
But if it uses a connection pool to talk to the DB, then there is (AFAICT)
no guarantee that the corresponding DB commands ("set A's location to 2," and
"set A's location to 1"), will arrive at the DB in the correct order. And even
if they did, the DB considers them to come from different clients (different
ports), so it may execute them in the opposite order. And because these
updates are asynchronous, the game loop has meanwhile moved on, having already
reported to other players that A moved to 2 and then back to 1. Even if such
problems might be avoidable by clever engineering of the client, the server
must not count on them. Instead it should enforce the consistency on its own,
by using only a single DB connection. (AFAICT,
such problems do not arise for a network application that is not required to
inform concurrent users of each others' activities, as in a vanilla web app,
which is why it can use a pool.)