Normally, one seeks to make a game engine faster.
But the near-term plan for this one is actually to make it slower---temporarily.
When performance is not the controlling criterion, as it currently is not,
the first priority of design becomes modularity.
And this engine is not yet as modular as it could be, because it interfaces
with the database directly, and hence exposes the choice of underlying DB
engine (Postgres), busies itself with talking to it,
and, as shown above, and may in future burden itself with the intricacies of
caching.
The code would be cleaner, smaller, and more focused, if this job was
instead outsourced to a separate data server (data access layer), which would
hide the choice
of underlying DB engine, take on the jobs of caching, conversion, and
enforcement of any integrity constraints (though one naturally tries to put
these in the DB engine itself), and with which the game engine would
communicate by REST API, JSON RPC, or whatever is appropriate.
By making it
wait for the data server, this modular separation will of course slow the game
engine down, especially if communication is via REST API. The separation will
also introduce the overhead of managing two separate code bases, along with all
that that entails: separate compilation, ensuring agreement of schema and
message format, the risk that the data server may not stay as ``fresh'' to the
developer as the game engine would, etc.

But the plan also paves the way for interesting gains, now and in future.
First there is the question of what language to use for the data server.
A glance at
\footlink{benchmarks}{http://www.techempower.com/benchmarks/} shows two
server frameworks that vastly outperform
most others: Rust's Actix, and JVM's Vert.x.\ The latter brings the possibility
of using Kotlin, arguably the nicest JVM language that is statically typed.
But unfortunately Kotlin does not teach any interesting concepts: although its
concrete syntax is better than Java's and Scala's, at the end of the day it is
just another object-oriented language that pays respect to the functional
paradigm. (I say this from experience: I have already used Vert.x + Kotlin to
implement a small financial web server.)

Rust is much more interesting.
Among its many strengths are that it belongs
quite fairly to the functional family of languages and yet compiles directly
to native code, producing a small and fast executable.
It has immutable
variables by default, enjoys a thriving ecosystem which includes such packages
as Tokio, Hyper, and Actix, offers zero-cost abstractions, beautiful ones in
both functional and asynchronous programming.
The result is more
beautiful than what can the JVM can produce, and typically faster. (For all its
speed after JIT-warmup, the JVM heaves and lumbers, while Rust remains nimble
and clever.)

But perhaps most importantly, Rust (AFAIK) is the famous inaugurator of linear
types, in the form of borrow checking.
This is a new paradigm for many developers, and for all its very friendly error
messages, the compiler still manages to be a hard taskmaster.
And even when it relents, compilation times can be long.
So the language takes time to master.
But it is worth it.
The language forces the programmer to learn sophisticated new skills.

So, the near-term plan is to factor database calls out of the Haskell game
engine and into a Rust data server, leaving the game engine to focus on its
main task of multiplexing communications.
This factoring-out can be done in a phased manner.
The data server can start off providing just one service endpoint, say the one
that, given the ID of a Thing, provides its name and description.
Once this is working, the Haskell engine in production can be easily modified
to use it and can continue to use Postgres directly for the other queries and
updates.
As the Rust engine grows to provide more of the data services, the Haskell
engine can continue to be gradually and easily modified to use them, even in
production.
Thus Postgres will eventually be completely factored out of the Haskell game
engine.
Since the game is currently only text-based and does not need the detailed
granularity of, say, a first-person shooter, the communication delays
introduced by this factoring out of the data access layer should be outweighed
by the gain in modular simplicity.
