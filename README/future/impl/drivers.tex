The engine should be extended to open a TCP port that allows authorized backend
components to connect and {\em drive} the game, by spawning NPCs and endowing
them with goals and behaviors.
Because such a driving module would manage many NPCs and
would multiplex the communication between them
and the engine, it would put its single TCP connection to better use than a
bot does.

The NPCs of such a module should react only
{\em via} the game engine.
If, for example, the game engine allows characters to
punch each other, and if a module decides to make one of its NPCs
punch another, the module must not try to simulate the punch internally.
Rather, it must send the game engine a request representing the punch, and
cannot conclude that the punch happened until it receives confirmation from the
game engine, in the form of a server message.

The connection between module and engine should carry JSON for now.
If and when
it becomes a bottleneck, an alternative TCP port can be opened for the same
purpose but communicating in raw binary.
But if the communication is dominated
by strings of text then this will not bring a big improvement, because in
strings there is already a one-to-one (or thereabouts) correspondence between
character and byte.

Driving modules can augment the game with elements and NPCs of any kind.
A module might supply the game with "sleep-sweepers," NPCs that
roam the world and return any sleeping character (derelict player) to its
home, so as not to "litter" the world with unresponsive characters.
But first
the (already partially implemented) asleep/awake functionality would have to
be completed.

Also possible is a driving module that provides driving---literally.
It could
fill the world with taxis that can be summoned and that take characters to
different places.
First, though, the game engine would need to be extended to
give a thing (a taxi) the ability to contain or carry another thing or player.
For the sake of realism, the ability to teleport is not planned.

These are only two possibilities.
More interesting ones are mentioned in the section on game play below.

Since the interface between a driving module and the game engine is just JSON
(or raw binary), the module can be implemented in any language, so there is
an opportunity here for polyglottism.
And since
\footlink{actors}{https://en.wikipedia.org/wiki/Actor\_model} can be used to
model NPCs, a driving module might be nicely implemented in Erlang, Actix,
or Akka, though Akka is apparently
\footlink{quite slow}{http://www.techempower.com/benchmarks/}.

A driving module can extend the capabilities of the game without necessarily
populating it with NPCs. For example a module can be added that provides
directions from any place in the game world to any other. It might construct
a graph of the world directly from the database (\ie, without going through
the game engine), and upon request it would run
a shortest-path algorithm between any two nodes (places), sending
the requested path back via the game engine as a JSON array.
But actually
this functionality might be better implemented as side-channel REST, provided
say by the Django server.

Driving modules should typically be dynamically tunable.
For example, one that fills the game with police agents should allow the game
administrator to adjust the density of such agents in the game world, while
the game is running.
The module can do this by exposing to the game
administrator a small REST API\@.